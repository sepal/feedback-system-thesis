\chapter{Introduction}

Getting feedback for a software system is one of the most important tasks, since it gives insights if the implemented system works, if there are any bugs and also what other features could be implement in order to improve the software.

\section{Motivation/Problem statement}

% todo: figure out how long aurora has been in use.
For the last \emph{n} years the learning platform Aurora has been used by hundreds(?) of students for several courses at the Vienna University of Technology. The system was originally implemented by other students for the master thesis or for courses(?).

A system created by so many different persons naturally introduces new bugs, which are hard to track, since often the not the original author, but a different student might have to fix it.

In the past an online notebook has been used to keep track of different issues and features request for the aurora system. Every year a new notebook was created and people could freely add bugs and feature requests to it. This method of getting feedback came at high cost though:

The notebook would get cluttered within the first weeks and keeping it organized, meant that someone had to constantly moderate the notebook.
People would often add the same issue, since it was hard to find existing issues.
As soon as an issue was fixed, the issue had to be found in the notebook and somehow marked as such. 


\section{Aim of work}

The aim of this thesis was to find a solution to collect feedback efficiently, without the high moderation costs that normal issue queues introduce and with ability to give users feedback about the state of their issue. Another requirement was that system should collaborative in order to encourage to the users to post, but also discuss feedback given.

The last requirement was to find a system that could be integrated directly into aurora. Using an external was not an option, since Austrian laws requires to keep the data within the infrastructure of the university, but also because requiring students to register at another platform would have added an additional barrier for the students to give feedback.